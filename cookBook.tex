% \documentclass[a4paper]{article}
\documentclass[]{report}

\usepackage{tgtermes}         % for the looks
\usepackage[T1]{fontenc}     % to hypenate accented characters correctly
\usepackage[utf8]{inputenc}  % just in case we go for non-ascii
\usepackage{tikz}            % the graphics
\usepackage{textcomp}        % for \textregistered
\usepackage{anyfontsize}
\usepackage{imakeidx}        % for index
\usepackage{hyperref}
\usepackage{titlesec}        % adjusting document class formatting
\usepackage{titletoc}        % adjusting document /tableofcontents
\usepackage{multicol}
\usepackage{changepage}
\usepackage[margin=1.5in]{geometry} % 1.75" is default

\definecolor{myColor}{RGB}{101,28,34} % Cover/accent color 

% \titleformat{\chapter}[block]
%   {\normalfont\huge\bfseries}{\thechapter.}{1em}{\Huge}

\titlespacing*{\chapter}{0pt}{-19pt}{0pt}

\dottedcontents{chapter}[4.5em]{\bfseries\normalsize\vspace{.75em}}{4.5em}{1pc} % \tableofcontents
\makeindex[options=-s dotted, columns=1] % https://latex.org/forum/viewtopic.php?t=22245
\indexsetup{othercode=\bfseries\normalsize\vspace{1em}}


\renewcommand{\thechapter}{\Roman{chapter}} % Forces roman numerals 

\titleformat{\chapter}[display] % chapter formating 
      {\bfseries\Large}
      % {\filleft\MakeUppercase{xxx\chaptertitlename} ttt\Huge\thechapter}
      {\filleft\Huge\thechapter}
      {0ex}
      {\color{myColor}\titlerule[3pt]\color{black}
      \vspace{1ex}%
      \filright}
      [\vspace{.5ex}%
      \color{myColor}{\titlerule[3pt]}]



%%%%%%%%%%%%%%%%%%%%%%%%%%%%%%%%%%%%%%%

% \newcommand{\meal}[5]{
% \chapter{\uppercase{#1}}  % recipe name
% \vspace{0.5cm}
% \textit{#2}\vspace{0.5cm} \\                      % description (if needed)
% {\textbf{Serves: #3}}                % serving size
% \vspace{0.5cm}\\
% \begin{minipage}[t]{0.5\textwidth}
%       #4                            % list of ingredients 
% \end{minipage}
% \begin{minipage}[t]{0.5\textwidth}
%       \vspace{-0.7cm}
%       \textbf{#5}                            % cooking instructions
% \end{minipage}
% }

\newcommand{\meal}[5]{
\chapter{\uppercase{#1}\index{\uppercase{#1}}}  % recipe name
\vspace{0.5cm}
\textit{#2}\vspace{0.5cm} \\                      % description (if needed)
{\textbf{Serves: #3}}                % serving size
\vspace{0.1cm}\\
\begin{adjustwidth}{-1.7em}{1.5em}
\begin{minipage}[t]{0.42\textwidth}
      #4                            % list of ingredients 
\end{minipage}
\begin{minipage}[t]{0.62\textwidth}
      \vspace{-0.7cm}
      \textbf{#5}                            % cooking instructions
\end{minipage}
\end{adjustwidth}
}

% \newcommand{\meal}[5]{
% \chapter{\uppercase{#1}}  % recipe name
% \vspace{0.5cm}
% \textit{#2}\vspace{0.5cm} \\                      % description (if needed)
% {\textbf{Serves: #3}}                % serving size
% \vspace{1cm}\\
% \begin{multicols}{2}
%       #4                            % list of ingredients 
% \end{multicols}
% \textbf{#5}
% }


%%%%%%%%%%%%%%%%%%%%%%%%%%%%%%%%%%%%%%%








\begin{document}

\thispagestyle{empty}

%%%%%%%%%%%%%%%%%%%%% Cover Page %%%%%%%%%%%%%%%%%%%%%

\begin{tikzpicture}[remember picture,overlay]

\node [align=center] at (current page.north) % upper
      {\begin{tikzpicture}[remember picture, overlay]
	  \fill[myColor] (-10,0) rectangle (10,-0.5);
	  \node[anchor=north] at (0,-0.5) {\fontsize{29}{34}\selectfont\textit{Recipes I want to keep track of}};
      \end{tikzpicture}};
      


\node [yshift=5cm] at (current page.center) % title
      {\begin{tikzpicture}[remember picture, overlay]
          \fill[myColor] (-9.5,-5) rectangle (9.5 ,1);
          \node[align=center] at (0,-2)
               {\resizebox{1.57\linewidth}{!}{\hspace{.25cm}\Huge\color{white}Food\hspace{2.1cm}}};
	       \node[anchor=east,align=right] at (1.2,-5.5) {\fontsize{29}{34}\selectfont\color{black}\textit{In no particular order}};
      \end{tikzpicture}};

\node [align=center, yshift=-6cm] at (current page.center) % animal image
      {\begin{tikzpicture}[remember picture, overlay]
          \node[anchor=center,align=center] at (0,5)
               {\includegraphics[width=1.65\textwidth]{tarsier3.png}};
      \end{tikzpicture}};

% \node [shift={(-5cm,-5cm)}] at (current page.north east) % Revision 2
%       {\begin{tikzpicture}[remember picture, overlay]
%           \draw[fill=black] (0,5) -- (1.5,5) -- (5,1.5) -- (5,0) -- cycle ;
%           \node[inner sep=0pt,rotate=-45] (rev) at (2.9,2.9) {\huge\color{white}\textbf{Revision 2}};
%       \end{tikzpicture}};

\node [align=center] at (current page.south) % bottom
      {\begin{tikzpicture}[remember picture, overlay]
          \node[anchor=south west, align=left] at (-10,0.5)
               {\resizebox{.3\linewidth}{!}
                 {\Huge\color{black}\fontfamily{phv}\selectfont
                   {O\color{myColor}'\color{black}RLY\textsuperscript{\color{myColor}\Large\textbf{?}}}}};
               \node[anchor=south east, align=right] at (10,0.5) {\Huge\color{black}\textit{Patrick McMackin}};
      \end{tikzpicture}};

\end{tikzpicture}

%%%%%%%%%%%%%%%%%%%%% Cover Page %%%%%%%%%%%%%%%%%%%%%


\tableofcontents

%%%%%%%%%%%%%%%%%%%%%%%%%%%%%%%%%%%%%%%%
      % \meal{RecipeTitle}
      %       {OptionalComment}
      %       {5--8}                % Serving Size
      %       {\begin{itemize}      % Ingredients
      %             \item 
      %       \end{itemize}}
      %       {\begin{enumerate}    % Cooking Instructions
      %             \item  
      %       \end{enumerate}}
%%%%%%%%%%%%%%%%%%%%%%%%%%%%%%%%%%%%%%%%


      \meal{Spicy-Sweet Sambal Pork Noodles}
            {From Bon Appetit}
            {6--8}
            {\begin{itemize}
                  \item 4 Tbsp. extra-virgin olive oil
                  \item 4 lb. ground pork, divided
                  \item 2" piece fresh ginger, peeled, finely chopped
                  \item 12 Garlic cloves, minced
                  \item 4 Tbsp. Sugar
                  \item 4 Tbsp. Tomato paste
                  \item 4 Sprigs basil, plus garnish
                  \item 2 Cup hot chili paste (such as sambal oelek)
                  \item 2 Cup soy sauce
                  \item 2 Cup unseasoned rice vinegar
                  \item 2 lbs ramen noodles
                  \item Salt/Pepper
                  \item 4 Tbsp. unsalted butter
                  \item 2-3 Broccoli crowns
            \end{itemize}}
            {\begin{enumerate}
                  \item  Heat oil in a large wide heavy pot over medium-high. Add half of pork, breaking apart with wooden spoon. Cook until well browned underneath, about 5 mins. Turn and continue to cook until pork is browned on 2--3 sides, about 5 mins.
                  \item Add ginger, garlic, sugar, and remaining pork and cook until meat is nearly cooked through, about 5 mins. Add tomato paste and 2 basil sprigs. Cook, stirring  occasionally, until paste darkens, about 2 minutes.
                  \item Add chili paste, soy sauce, vinegar, and 2 cups water. Bring to a simmer, reduce heat to low, and cook, uncovered and stirring occasionally, until sauce is slightly thickened and flavors have melded, 30—45 minutes.
                  \item Dice broccoli finely, add to pot in last about 10 minutes of cooking
                  \item Cook noodles in a large pot of boiling salted water, until 1 minute short of al dente. Add to sauce with butter and a splash of pasta cooking liquid. Simmer, toss, until sauce begins to cling to noodles, about 1 minute. Add basil.
            \end{enumerate}}
            
      \meal{Halal Chicken and rice}
            {From Ethan Chlebowski, optionally add red sauce (or make chicken spicy). Season aggressivly and sear the chicken hard. }
            {6--8}                % Serving Size
            {\begin{itemize}      % Ingredients
                  \item 4 lbs skinless Chicken thighs
                  \item Salt/Pepper
                  \item Cayenne
                  \item 5 whole cloves, crushed
                  \item Cumin
                  \item Oregano
                  \item 3 Garlic cloves
                  \item Lemon juice
                  \item Mayo
                  \item 2 Tbsp Butter 
                  \item 1/2 Onion, diced
                  \item Turmeric 
                  \item Smoked Paprika 
                  \item 1 Bay leaf
                  \item 2 cups Basmati rice
                  \item 2 cups water or stock 
                  \item plain Greek yogurt
                  \item 10 g white Vinegar
            \end{itemize}}
            {\begin{enumerate}    % Cooking Instructions
                  \item Marinated Chicken: Salt the chicken thighs and set them aside. Crush the cloves and cumin in a mortar and pestle. Add oregano, black pepper, cayenne, and garlic cloves into the mortar with the spices and crush into a rough paste. In a large mixing bowl, combine the lemon juice, mayo, and spice mixture. Add the chicken thighs and thoroughly coat the exterior. Cover and place in the fridge for up to 24hrs or cook right away.
                  \item Place a pan over high to medium-high heat. Once hot, sear the chicken thighs until deeply browned. Chop the chicken into pieces in the pan
                  \item Yellow Rice: Melt butter in a pot over medium heat. Add onion, cumin, turmeric, smoked paprika, pepper, and bay leaf. Stir aromatics together until fragrant but not burnt, about 30 seconds. Add the rice to the pan with the aromatics and mix. Lightly toast the rice and stir for about 2 minutes. Add chicken broth, turn up the heat, and cover the pan to bring to a boil. Turn the heat to the lowest setting. Let rice steam covered for about 20 mins
                  \item Mix white sauce (2 parts mayo, 1 part yogurt, oregano, lemon juice, vinegar, salt, paprika, MSG, garlic powder) in bowl to taste, drizzel over chicken and rice. Serve with simple lettuce/tomato/onion salad and naan
            \end{enumerate}}

      \meal{Chicken Curry}
            {Basic. Optionally add MSG, lemon grass, curry powder, ect}
            {5--8}
            {\begin{itemize}
                  \item 3 cups (dry) rice (cooked)
                  \item 4lbs chicken thighs
                  \item Cinnamon
                  \item Tumeric
                  \item Garam Masala
                  \item 1 Hot chili (diced)
                  \item 2 Bell peppers (diced)
                  \item 2 Large onion (diced)
                  \item 2" knob ginger (ground to paste)
                  \item 5--6 Cloves garlic (ground to paste)
                  \item Ground cardamom
                  \item Chili powder
                  \item Cayenne powder
                  \item Diced tomatoes (2x 28 oz cans)
                  \item 1--2 Cups stock
                  \item Coconut milk (2-3 140z cans)
                  \item Bay leaves
                  \item Diced coriander
                  \item Salt/pepper
            \end{itemize}}
            {\begin{enumerate}
                  \item Marinade chicken with oil, salt, 1 tbs tumeric, 1 tbs garam masala, 1/4 tps cinnamon
                  \item Brown chicken in batches. (No need to cook all the way)
                  \item Remove chicken, add chili, peppers, onion, garlic, ginger. Cook
                  \item Add 1/4 tps ground cardamom, 1 ths garam masala, 1 tbs chili powder, cayenne powder, salt, pepper and bay leaves. Bloom spices. (Add spices generously, and "by the heart")
                  \item Return chicken to pot, add tomatos, stock
                  \item Simmer about 30 minutes without lid
                  \item Add coconut milk, cook another 15--20 mins
                  \item Mix in diced coriander, serve over rice
            \end{enumerate}}






      \meal{Slow cooker chicken tacos}
            {Dead easy}
            {6--8}
            {\begin{itemize}
                  \item Chicken breast (4--5 lbs)
                  \item Black beans drained (28oz can)
                  \item Corn drained (14oz can)
                  \item Taco seasoning package
                  \item Ranch seasoning package
                  \item Adobo peppers (2-3)
                  \item Crushed/diced tomatoes (28oz can)
                  \item Cream cheese (3/4 package)
                  \item Cilantro
                  \item Salt/Pepper + whatever spices (Cumin...)
            \end{itemize}}
            {\begin{enumerate}
                  \item Slowcook 3 hours on high
                  \item Remove and shred chicken
                  \item Return chicken, add chopped cilantro
                  \item Serve on wraps or with rice
            \end{enumerate}}


      \meal{Beef Chili}
            {From J Kenji Lopez-Alt}
            {6--8}
            {\begin{itemize}
                  \item 3lbs Ground beef
                  \item 2 Onions
                  \item 6ish Cloves garlic
                  \item 6ish Anchovy fillets
                  \item 1 small can of Peppers in adobo
                  \item 1 Tbsp Oregano
                  \item 1 Tbsp Cumin
                  \item 2 Tbsp Chili powder 
                  \item 1/2 cup Tomato paste
                  \item 28oz can Tomato (whole or chunk)
                  \item 28oz can Black beans (or kidney)
                  \item 2ish cups Frozen corn
                  \item 1-2 cups Chicken stock
                  \item 3 thps Instant cornmeal (maseca)
                  \item 1 tbps Cocoa powder
                  \item MSG
                  \item Worcestershire sauce
                  \item Soy Sauce
                  \item 1 tbps Brandy
                  \item 3 tbps Hot sauce
                  \item Salt/Pepper
            \end{itemize}}
            {\begin{enumerate}
                  \item Brown beef with oil fairly hard (in batches). Add diced onions to the end to soften, add garlic in last 60 seconds of cooking
                  \item Mash anchovy fillets with fork, dice peppers and add all ingredients to slowcooker on low for 8 hours (include adobo sauce). Salt/pepper
                  \item Serve with garnishes (greens, cheese, sour cream, cornbread ect)
            \end{enumerate}}


      \meal{Enchiladas Verde con pollo}
            {From Adam Ragusea}
            {5--7}
            {\begin{itemize}
                  \item 4 lbs chicken thighs (skinless)
                  \item 2 large White onions
                  \item 1--2 dozen Tomatillos
                  \item White wine
                  \item Cumin
                  \item Coriander
                  \item Paprika
                  \item Salt/Pepper
                  \item 4--8 Jalpe\~nos
                  \item 2 Limes
                  \item 4--5 Garlic cloves
                  \item Cilantro
                  \item 8oz Monterey Jack cheese
                  \item 16-20 7" Tortillas
            \end{itemize}}
            {\begin{enumerate}
                  \item  Fry the chicken thighs in olive oil until browned, and remove.
                  \item Roughly chop one of the onions and fry it in the same pot until golden. Put in maybe a teaspoon (or more) each of the ground cumin,               coriander and paprika, grind in some pepper, and fry the spices for a minute. Put the chicken back in, and cover with about half white wine, half water. Salt. Cover, reduce heat to a simmer, and cook until the chicken begins falling about (30 mins).
                  \item For the salsa verde, cut the chiles in half, scrape out and discard the seeds. Husk the tomatillos, rinse, and cut them in half. Roughly cut other onion. Put the veggies and lime into a wide pan and toss them in olive oil. Put the pan into the oven at 450 F. After the salsa has a bit of color, put in the whole peeled garlic cloves and stir. Cook maybe 20 mins total. Broil for a few minutes on high, Squeeze the lime juice and discard. Puree until smooth.
                  \item Grate up the cheese. Put maybe half a cup of the salsa into the chicken, along with half of the cheese. Mix up the filling and check for seasoning.
                  \item Grease a 7x11 inch baking dish. Dunk a tortilla into the salsa, scoop in a modest line of filling, and roll up the tortilla, seam-side down. Repeat until you've filled your baking dish and/or exhausted your filling. Spoon the extra salsa and cheese on top of the enchiladas. Bake uncovered until the cheese is browned, maybe 20 mins. Garnish with cilantro.
            \end{enumerate}}


      \meal{Peanut chicken bake}
            {Personal Favorite, found on Reddit}
            {5}
            {\begin{itemize}
                  \item 10 bone in, skin on Chicken thighs
                  \item 5--10 Chilis (anything hot)
                  \item 1 large Naval orange
                  \item 2--3 large Onions
                  \item 3/4 cup natural Peanut butter
                  \item 1 cup roasted Peanuts
                  \item Cilantro
                  \item 1/2 cup Soy sauce
                  \item 1/3 cup Fish sauce
                  \item 2-3 tbs Apple cider vinegar
                  \item 1/3 cup Brown sugar
                  \item 1 tbs Cumin
                  \item MSG
                  \item 2" knob Ginger
                  \item 5--7 cloves Garlic
                  \item White rice
                  \item Side veggie (Roast asparagus)
            \end{itemize}}
            {\begin{enumerate}
                  \item  Brown chicken skin side down on stove at medium heat until golden, flip to other side for 1-2 minutes and set aside (should not be fully cooked yet)
                  \item In pot, cook sliced onion and chillis with salt and oil, add soy sauce, peanut butter, orange juice and zest, fish sauce, MSG, vinegar, cumin, grated ginger and garlic. Cook until smooth
                  \item In baking dish, pour sauce, then add chicken thighs on top. Skin should be exposed. Bake at 400 for 25-30 minutes or until done
                  \item Garnish with cilantro and diced peanuts, serve with over rice with side veggies
            \end{enumerate}}



      \meal{Baked chicken wings}
            {From J Kenji Lopez-Alt}
            {5}
            {\begin{itemize}
                  \item 3 lbs Chicken wings
                  \item 3 tps Baking powder
                  \item 3 tps Salt
                  \item 3 tps Cornstarch
                  \item 1 cup butter
                  \item 1 cup Frank's Red Hot
            \end{itemize}}
            {\begin{enumerate}
                  \item Add teaspoon of salt, cornstarch, baking powder to every pound of wings
                  \item Let sit skin-side up on baking sheet for a few hours (uncovered) in fridge
                  \item Roast at 450F for 20 mins
                  \item Flip, continue cooking for 15-25 mins (maybe flip more)
                  \item Melt 4tbs butter with 4 tbls Frank's Red Hot (can add garlic)
                  \item Toss with wings in bowl
            \end{enumerate}}



      \meal{Pickled Onions}
            {Add to anything, keeps for a week or two in fridge}
            {0}                % Serving Size
            {\begin{itemize}      % Ingredients
                  \item Red onions
                  \item 1 part Vinegar
                  \item 1 part Water
                  \item Big pinch of Salt
                  \item Big pinch of Sugar
            \end{itemize}}
            {\begin{enumerate}    % Cooking Instructions
                  \item  Very thinly slice the red onions from root to stem and add them to a jar.
                  \item Add equal parts vinegar and water to a pot (enough to cover the onions) with a big pinch of salt and sugar. Bring to a boil and pour over the onions.
                  \item Cover and let sit in the fridge for at least 2 hours before serving. The pink color will get deeper the longer they sit.
            \end{enumerate}}


      \meal{Honey Sriracha Meatballs}
            {Mealprep with couscous and mixed frozen veggies}
            {6--8}                % Serving Size
            {\begin{itemize}      % Ingredients
                  \item 4 lbs ground Meat (beef+pork works best)
                  \item 2 cups panko Breadcrumbs
                  \item 4 Eggs
                  \item 1/2 cup green Onion
                  \item 1 tsp. Garlic powder
                  \item Salt/Pepper
                  \item 1/2 cup Sriracha
                  \item 6 Tbsp Soy Sauce
                  \item 6 Tbsp Honey
                  \item MSG
                  \item 6 Tbsp Rice vinegar
                  \item 2" knob grated Ginger
                  \item 6 cloves miced Garlic
                  \item 1 tsp. Sesame oil
                  \item Tbsp Cornstarch
            \end{itemize}}
            {\begin{enumerate}    % Cooking Instructions
                  \item  In a large bowl, mix together turkey, breadcrumbs, eggs, green onions, garlic powder and salt/pepper until well combined. Shape mixture into 1 1/2-inch balls and place spaced apart on prepared baking sheets lightly sprayed with cooking spray.
                  \item Bake meatballs for 20 to 25 minutes at 375 F, broil last 2 mins
                  \item Combine all the ingredients for the sauce in a small saucepan and bring to a boil over medium heat, whisking continuously. Reduce heat and simmer for 8 to 10 minutes (the sauce will start to thicken) then toss with the meatballs.
                  \item Garnish with sesame seeds and green onion, serve over carbs
            \end{enumerate}}



      \meal{Curried Cauliflower, refried beans \& Chicken}
            {A bastardization of cultures but I like it}
            {6--7}                % Serving Size
            {\begin{itemize}      % Ingredients
                  \item 5 lbs skinless Chicken thigh
                  \item Basic marinade (whatever is cheap from the store)
                  \item Pinto Beans (28oz can)
                  \item 3-5 Poblano peppers
                  \item large Onion
                  \item Cayenne
                  \item Salt/Pepper
                  \item Cumin
                  \item 3 heads Cauliflower
            \end{itemize}}
            {\begin{enumerate}    % Cooking Instructions
                  \item  Chop Cauliflower, toss in oil, salt, cayenne and curry powder. Roast at 425 for 25-35 minutes
                  \item Marinade chicken for at least one hour, grill
                  \item Dice onion and peppers, fry in large pan until soft. Add cumin, salt, cayenne and toast for 30 seconds.
                  \item Push veggies to the side, add drained can of beans on low heat and mash  with large fork until mostly smooth
                  \item Mix beans and veggies together, season to taste
            \end{enumerate}}


      \meal{Slowcooked Asian Chicken}
            {Not traditional in any sense}
            {5--7}                % Serving Size
            {\begin{itemize}      % Ingredients
                  \item 4 lbs Chicken breast
                  \item 1 cup Soy sauce
                  \item 1 cup Honey
                  \item 1/2 cup Hoisin sauce
                  \item 2 Tbsp rice wine vinegar
                  \item 2 tsp Sesame oil
                  \item 2" Ginger, grated
                  \item 5--8 cloves Garlic, miced
                  \item 4 Tbsp Sriracha
                  \item Green Onion, sliced
                  \item 2 Tbsp Cornstarch
                  \item Sesame seeds
                  \item Rice or noodles
            \end{itemize}}
            {\begin{enumerate}    % Cooking Instructions
                  \item  In a medium bow whisk together soy sauce, honey, hoisin, vinegar, sesame oil, ginger, garlic, sriracha and the white parts only of the scallions
                  \item Pour the sauce mixture ontop of the chicken and slowcook for 2--3 hours on low. (Cooking time depends on the thickness of your chicken.)
                  \item Remove the chicken from the liquid and place on a plate or cutting board. Allow the chicken to rest for a couple of minutes and then shred with two forks or slice it up with a knife.
                  \item Transfer the liquid from the slow cooker to a small saucepan. Whisk in cornstarch slurry. Cook on high until bubbly and thickened, whisking constantly. Pour the sauce back in the slow cooker along with the shredded chicken and toss to coat.
                  \item Serve immediately over rice or noodles with sesame seeds and green onions for garnish.
            \end{enumerate}}



      \meal{Jambalaya}
            {Cook the roux as dark as possible, do not walk away and burn it. From Isaac Toups}
            {5--8}                % Serving Size
            {\begin{itemize}      % Ingredients
                  \item 1 lbs Andouille sausage, sliced
                  \item 1--2 lbs boneless Chicken thighs
                  \item Scallions
                  \item 2 lbs ground Beef
                  \item Cumin
                  \item MSG
                  \item Salt/Pepper
                  \item Cayenne
                  \item 1 cup Vegetable oil
                  \item 1 cup Flour
                  \item 1 bag Celery
                  \item 3--4 Bell Peppers
                  \item 2--3 large Onions
                  \item Bay leaves
                  \item 5--8 cloves Garlic, crushed
                  \item 1 dark Beer 
                  \item 2--4 cups Stock 
                  \item cooked White Rice (3 cups dry)
            \end{itemize}}
            {\begin{enumerate}    % Cooking Instructions
                  \item Dice holy trinity (onion, peppers, celery), heat 1 cup oil on medium high. Add flour, mix well. Whisk continuously until the roux is the color of a chocolate bar (10--20 mins). 
                  \item Add holy trinity and bay leaves. Cook for a few mins, then add garlic. Then slowly whisk in beer and maybe 1 cup stock. Remove from heat.
                  \item  In seperate large pan a generous amount of fresh black pepper as well as cumin and cayenne (to taste) to ground beef. Once both sides are seared, give it a good chop. Add to pot of roux, deglass if needed
                  \item Add chicken stock, sausage, let simmer on low for about 45 mins. Meanwhile, sear chicken in pan (or grill), add to pot in last 10 minutes of cooking 
                  \item Add cooked white rice, garnish and eat
            \end{enumerate}}



      \meal{Habanero Chimichurri pasta salad}
            {Can replace pasta with something like farro, from Ethan Chlebowski}
            {6--8}                % Serving Size
            {\begin{itemize}      % Ingredients
                  \item 4--5 lbs boneless Chicken thighs
                  \item 2 lbs dried Pasta
                  \item Cilantro
                  \item 2 orange Bell Peppers 
                  \item 4--5 Onions
                  \item 2 containers cherry Tomatoes 
                  \item 3 Habaneros 
                  \item 2--3 Limes 
                  \item 16 oz block Feta 
                  \item 2 cloves Garlic 
                  \item 5--6 Poblano Peppers 
                  \item Pickled Onions
            \end{itemize}}
            {\begin{enumerate}    % Cooking Instructions
                  \item  Slice 4ish onions into rings, and poblanos into large chunks. Toss in oil, salt/pepper. Grill until charred.
                  \item Simply season chicken, add oil (or mayo) and grill. Slice 
                  \item Add 2 lbs pasta to boiling water and cook as directed. Meanwhile, slice cherry tomatoes in half and add to large mixing bowl 
                  \item To blender, add cilantro, habaneros (deseeded), garlic, olive oil, 1/2 onion, bell peppers and lime juice. Blend until smooth, season to taste 
                  \item Toss the pasta in the mixing bowl with 3/4's of the chimichurri. Add grilled onions, peppers, chicken and crumbled feta 
                  \item Serve hot or cold with pickled onion and extra chimichurri on top 
            \end{enumerate}}

      \meal{Shrimp and Grits}
            {Cajun seasoning works well, from Babish}
            {5--7}                % Serving Size
            {\begin{itemize}      % Ingredients
                  \item 2 lbs Shrimp with shell
                  \item smoked Paprika
                  \item Oregano
                  \item Garlic powder
                  \item Cayenne
                  \item Bay leaf(s)
                  \item green Bell Pepper
                  \item Onion 
                  \item 2--3 green Onions 
                  \item 3 gloves Garlic 
                  \item Lemon juice 
                  \item 6 Tbsp Butter 
                  \item 1.5 cups Grits 
                  \item 3 cups Stock 
                  \item 3 cups Milk 
                  \item 1/2 cup Parmesan cheese, grated 
            \end{itemize}}
            {\begin{enumerate}    % Cooking Instructions
                  \item  Combine stock, milk, and salt in a medium pot. Add the grits to the boiling liquid while whisking. Reduce the heat to low and simmer the mixture for 15--25 minutes. Stir occasionally
                  \item Turn off the heat. Add the butter and cheeses to the grits and stir to combine them thoroughly. Season the grits to taste with more salt and pepper.
                  \item Peel and devein the shrimp. Combine the reserved shrimp shells in a medium pot with the oil. Sear the shrimp shells for 1--2 minutes, then add the water, peppercorns, and bay leaf. Bring to a boil and cook until the water has reduced to 1 cup of liquid. Strain the stock
                  \item Cook the bacon in a large high-sided skillet. Remove the bacon from the pan and chop into bite size pieces. Reserve
                  \item Sear the shrimp for 30 seconds on each side in bacon fat. Transfer the par-cooked shrimp to a separate plate. Sprinkle shrimp with spice mixture (salt, pepper, paprika, oregano, garlic powder, and cayenne)
                  \item Add onion, bell pepper, and green onion to the pan. Cook for 3--4 mins. Add garlic, cook another minute 
                  \item Deglaze the pan with the reserved shrimp stock. Simmer the vegetables for 3--5 minutes or until the liquid has reduced to about 1/2 cup. 
                  \item Add the shrimp back to the pan and cook until the shrimp is heated through, about 2 min. Finish the sauce with the lemon juice, tabasco sauce, and butter. Season to taste with more salt and pepper.
            \end{enumerate}}


      \meal{Red Beans and Rice}
            {Easy slowcooker meal, serve over white rice}
            {7--9}                % Serving Size
            {\begin{itemize}      % Ingredients
                  \item 1 bag Celery
                  \item 3--4 Bell Peppers
                  \item 2--3 large Onions
                  \item 4 lbs bulk Sausage 
                  \item 7 cups Water 
                  \item 1 lbs dried Red Beans 
                  \item Frank's Red Hot
                  \item Salt/Pepper 
                  \item 3 Tbsp Garlic powder
                  \item 3 Tbsp Onion powder 
                  \item 3--6 Bay leaves 
                  \item 3 tps dried Thyme
                  \item 1/2 cup Flour 
                  \item cooked White Rice 
            \end{itemize}}
            {\begin{enumerate}    % Cooking Instructions
                  \item Sear sausage hard in batches, break up into rough chunks 
                  \item Meanwhile, finely dice holy trinity. Remove sausage from pan and cook trinity until mostly dry (10--15 mins)
                  \item Optionally add flour (plus some oil) and cook out to make roux 
                  \item Trasnfer all ingredients to slowcooker. Cook for 8 hours on low, breaking up beans in last hour of cooking 
                  \item Taste for seasoning, serve over cooked white rice 
            \end{enumerate}}



      \meal{Cream of mushroom beef stroganoff}
            {Stick blender is useful here. Cream base from Adam Ragusea}
            {5--8}                % Serving Size
            {\begin{itemize}      % Ingredients
                  \item 4 lbs Stew beef (chunked chuck)
                  \item 2 packages egg noddles
                  \item 2 lb (454g) fresh mushrooms
                  \item 2 oz (30g) dried mushroom
                  \item Onion 
                  \item 2 cup (237mL) brandy/white wine 
                  \item 8 cups Water 
                  \item 1--2 cup (237mL) Cream 
                  \item 3--4 garlic cloves
                  \item 1/2 cup Flour 
                  \item 4 Tbsp Worcestershire
                  \item Butter
                  \item Salt/Pepper/MSG 
                  \item Parsley/Tarragon 
            \end{itemize}}
            {\begin{enumerate}    % Cooking Instructions
                  \item  Cook diced onion w/ olive oil in pot. Then melt 1/2c butter and make roux with flour, cook a few mintues
                  \item Deglaze with brandy or wine, cook out. Then gradually add the water
                  \item Add dried mushrooms, worcestershire, pepper, thyme, and simmer for 20--30 mins
                  \item In seperate pan, brown off beef in batches. Then cook the sliced fresh mushrooms (and garlic), set aside and deglaze if needed
                  \item Transfer mushroom stock from pot to blender, blend until smooth
                  \item Transfer back to pot, add beef and mushrooms, simmer for about 2 hours
                  \item Mix in cream, salt (vinegar if needed), serve over egg noodles, garnish
            \end{enumerate}}


      \meal{Biscuits and Sausage Gravy }
            {Calorie bomb, from J Kenji Lopez-Alt}
            {5}                % Serving Size
            {\begin{itemize}      % Ingredients
                  \item 1 cup Flour
                  \item 4.5 tsp (15g) Baking powder
                  \item 3/4 tsp (12g) Salt 
                  \item 15oz (2 cups) Heavy Cream
                  \item 12oz breakfast Sausage 
                  \item 3 Tbsp Flour (for gravy)
                  \item 3 cup Heavy Cream (or milk)
                  \item Salt/Pepper
            \end{itemize}}
            {\begin{enumerate}    % Cooking Instructions
                  \item  Preheat an oven to 425F. Combine the flour, baking powder, and salt in a bowl. Add the heavy cream and mix into a shaggy dough. Do not knead it. Dump it onto a floured work surface and for it into a rough 9x9" rectangle. Cut it into three biscuits with a sharp knife. Transfer to a baking sheet and bake until golden brown, about 15 minutes. (You can brush them with heavy cream or melted butter before baking for darker and more even browning, if you wish.)
                  \item Meanwhile make the gravy: cook the sausage in a medium skillet over medium-high heat, breaking it up with a wooden spoon until it’s not longer pink. If the pan is very dry, add a couple teaspoons of oil or butter to moisten it. Add 1 tablespoon of flour and stir to incorporate. Whisk in the heavy cream or milk. Bring to a simmer while stirring frequently and adjust the consistency with more heavy cream or milk if it’s too thick. Season aggressively with black pepper and salt.
                  \item When the biscuits are done, transfer to a plate and smother with the sausage gravy. Serve.
            \end{enumerate}}


      \meal{Christmas Chex Mix}
            {Do not overheat the chocolate}
            {5}                % Serving Size
            {\begin{itemize}      % Ingredients
                  \item 4 cups corn and/or rice Chex
                  \item 2 cups mini Pretzels
                  \item 1 cup roasted nuts
                  \item 1 Tbsp Butter 
                  \item 1/4 cup Honey
                  \item 1/4 cup Sugar 
                  \item 1 tsp Cinnamon
                  \item 1 tsp Vanilla extract
                  \item 1/2 cup white Chocolate chips
                  \item 2 Tbsp Coconut oil 
                  \item 1 cup red and green M\&M's
            \end{itemize}}
            {\begin{enumerate}    % Cooking Instructions
                  \item  Preheat the oven to 250F. Line a rimmed baking sheet with parchment paper. Combine the cereal, pretzels and nuts in a large bowl.
                  \item Melt the butter with the honey and sugar in a small saucepan over low heat until the mixture is bubbling and well combined, about 2 minutes. Add the cinnamon and vanilla and stir until well combined. Pour half of the butter mixture over the cereal mixture and toss until well coated. Add the remaining butter mixture and toss again until evenly coated.
                  \item Spread the mixture on the baking sheet in an even layer. Bake, stirring every 15 minutes, until toasted and just starting to dry out, about 1 hour.
                  \item Combine the white chocolate chips and coconut oil in a small microwave-safe bowl. Microwave at 50 percent power in 30-second intervals, stirring between each interval, until melted and smooth.
                  \item Drizzle the white chocolate over the snack mix on the baking sheet, then immediately sprinkle with the M\&M's. Let cool completely so the white chocolate sets, about 30 minutes. Break into small clusters.
            \end{enumerate}}

      \meal{Caesar Salad}
            {From Bon Appetit}
            {5--8}                % Serving Size
            {\begin{itemize}      % Ingredients
                  \item 18 Anchovy Fillets, drained
                  \item 3 Garlic clove
                  \item Salt/Pepper 
                  \item 6 large Egg Yolks
                  \item 6+ Tbsp Lemon juice
                  \item 2-3 tsp Dijon Mustard
                  \item 6 Tbsp Olive oil
                  \item 1.5 cups Vegetable Oil 
                  \item 9 Tbsp Parmesan, grated 
                  \item 1 loaf Sourdough, torn into chunks 
                  \item 9 Tbsp Olive oil
                  \item 8--9 Romain hearts, leaves seperated 
                  \item 4--5 lbs grilled Chicken
            \end{itemize}}
            {\begin{enumerate}    % Cooking Instructions
                  \item  Chop together anchovy fillets, garlic, and pinch of salt. Use the side of a knife blade to mash into a paste, then scrape into a medium bowl. Whisk in egg yolks, 2 Tbsp lemon juice, and mustard. Adding drop by drop to start, gradually whisk in olive oil, then vegetable oil; whisk until dressing is thick and glossy. Whisk in Parmesan. Season with salt, pepper, and more lemon juice, if desired. (Can be made 1 day ahead.)
                  \item Make your own crutons. Tearing, not cutting the bread ensures nooks and crannies that catch the dressing and add texture. Preheat oven to 375°. Toss bread with olive oil on a baking sheet; season with salt and pepper. Bake, tossing occasionally, until golden, 15--20 minutes.
                  \item Garnish salad with parmesan using a vegetable peeler to thinly shave a modest amount on top for little salty bursts.
                  \item  Use your hands to gently toss the lettuce, croutons, and dressing, then top off with the shaved Parm
            \end{enumerate}}


      \meal{KOREAN BEEF}
            {Classic easy mealprep}
            {6--7}                % Serving Size
            {\begin{itemize}      % Ingredients
                  \item 3 cups dry White rice, cooked
                  \item 4--5lbs ground Beef 
                  \item 4 cloves Garlic
                  \item 1/2 cup Brown Sugar 
                  \item 1/2 cup Soy Sauce
                  \item 1/2 cup rice Vinegar
                  \item 4" knob Ginger 
                  \item 4 tsp Sesame oil 
                  \item 4 Tbsp Sriracha
                  \item 2--4 Tbsp Cornstarch
                  \item MSG
                  \item Green onions 
                  \item 1 large Onion 
                  \item Sesame seeds 
                  \item Frozen peas (or mixed veggies)
            \end{itemize}}
            {\begin{enumerate}    % Cooking Instructions
                  \item  In a small bowl, whisk together brown sugar, soy sauce, ginger, sesame oil, MSG and Sriracha. Grate in garlic and half of the ginger. Thinnly slice remaining ginger 
                  \item Cook beef in large pan on high, draining as needed. Add minced onion when nearly cooked. 
                  \item Mix cornstarch with small amount of water or soy sauce, add to sauce. Add sauce to beef and allow to bubble and thicken. Adjust to taste
                  \item Serve over white rice, garnish with green onion and sesame seeds 
            \end{enumerate}}



      \meal{Clam Chowder}
            {From J Kenji Lopez-Alt}
            {6--8}                % Serving Size
            {\begin{itemize}      % Ingredients
                  \item 2 Onions (or leeks)
                  \item 4 stalks Celery
                  \item 4 Bay leaves 
                  \item 1 cup Flour
                  \item 2 cups Clam juice
                  \item 3--4 small cans diced Clams 
                  \item Butter 
                  \item Bacon 
                  \item 2--3 cups forzen Corn 
                  \item 1--2 russet Potatoes
                  \item 8 cups Milk
                  \item 2 Tbsp Fish sauce
                  \item Thyme 
                  \item Salt/Pepper 
            \end{itemize}}
            {\begin{enumerate}    % Cooking Instructions
                  \item  Cook bacon until crisp, remove. Keep fat in pot. (If no bacon, use butter). Finely dice onion and celery, cook in fat. Do not brown. Add pinch of salt
                  \item Peel and dice potatoes, cook until slightly tender in seperate pot of salted water
                  \item Add flour to main pot, stir and cook raw flour taste out. Stir in clam juice, and then slowly stir in 8 cups of milk
                  \item Add bay leaves, fish sauce, thyme and canned clams (juice and all). Bring to a simmer and cook about 10--15 mins.
                  \item Add potatoes for last 5 mins of cooking. Add salt and lots of black pepper
                  \item Serve with toast or oyster crackers
            \end{enumerate}}



      \meal{Ragu Bolognese}
            {From J Kenji Lopez-Alt}
            {6--9}                % Serving Size
            {\begin{itemize}      % Ingredients
                  \item 8--12oz Pancetta 
                  \item 1 lbs ground Beef
                  \item 1 lbs ground Lamb
                  \item 1 lbs ground Pork
                  \item 2 large Onions, diced 
                  \item 4 ribs Celery, diced 
                  \item 2 large Carrots, shredded 
                  \item 5--8 gloves Garlic 
                  \item Sage, thyme, rosemary, basil
                  \item 1/2 cup Tomato paste 
                  \item 2 cups White Wine 
                  \item 4 cups Chicken stock 
                  \item 2 Tbsp Gelatin 
                  \item 2 cups Milk 
                  \item Parmesan
                  \item 2 lbs dry Pasta
            \end{itemize}}
            {\begin{enumerate}    % Cooking Instructions
                  \item   Heat the oil in a wide straight-sided sauté pan or Dutch oven over medium-high heat until shimmering. Add the pancetta and cook, until it's well-browned and the fat has mostly rendered off, a few mins. Add the meat, season lightly with salt and pepper, and cook, breaking it up with a spoon, until it's pretty well-browned as well, 7 mins
                  \item Add the onion, celery, carrot, and garlic and cook, stirring, until softened but not browned, about 5 mins. Add the minced parsley and/or sage (or the bundle of rosemary or thyme) and the tomato paste and cook, stirring, until fragrant. There should be a large amount of browned solids on the bottom of the pan by now.
                  \item Add the wine and cook, scraping up the browned bits with a wooden spoon. Continue cooking until sauce is thick and the wine has fully reduced. Add the stock and milk.
                  \item Bring to a boil, then reduce heat until the sauce is at the barest simmer. Cook about 30 mins, stirring now and then, until the sauce is rich and thick and emulsified. Fat may break out as it cooks and forms a reddish slick on top. This is OK, just keep cooking and that fat will eventually get re-emulsified into the sauce.
                  \item  To serve, cook up some good fresh pasta (tagliatelle or rigatoni) just until al dente. Transfer the cooked pasta to a large skillet and spoon some of the ragu on top of it. Add some minced parsley and/or basil, some freshly grated cheese, a drizzle of olive oil, and a big splash of the pasta cooking-water. Cook over high heat, tossing constantly, until the sauce is creamy and coast the pasta. Serve right away.
            \end{enumerate}}


      \meal{Beef Short ribs}
            {Use english-cut ribs, from Adam Ragusea}
            {5}                % Serving Size
            {\begin{itemize}      % Ingredients
                  \item 5 lbs beef Short ribs
                  \item 1 large red Onion 
                  \item 3 Carrots 
                  \item 2 stalks Celery 
                  \item 2 Tbsp Tomato paste 
                  \item 1 bottle White Wine 
                  \item 1 star Anise 
                  \item 1 tsp Coriander Seeds 
                  \item Salt/Pepper 
                  \item Vinegar
                  \item 4 lbs Potatoes
                  \item Milk 
                  \item Rosemary
                  \item Frozen Peas 
                  \item Butter 
            \end{itemize}}
            {\begin{enumerate}    % Cooking Instructions
                  \item  Put a little oil in a large lidded pot on moderate heat and slowly brown the short ribs on all sides in batches. Do not let anything burn. Take them them out, and throw in the roughly chopped onion, carrots and celery. Let the veggies brown for a minute, then stir in the tomato paste. Then deglaze with the wine. Put in the anise, coriander, pepper and a big pinch of salt. Put the ribs back in, reduce the heat to a bare simmer, cover, and let cook for 8 hours.
                  \item Carefully remove meat to a plate, discard bones. Strain the braising liquid and discard the solids. Add some ice cubes and throw the liquid in the fridge to solidify fat out. Cover the meat and throw it in the fridge.
                  \item After you've de-fatted your braising liquid, reserve the fat and throw the liquid into a wide pan. Bring it to a boil and reduce it to a glaze — about a half hour. Stir it constantly and maybe reduce the heat toward the end to keep it from sticking and burning. Taste the glaze, and give it some more seasoning and maybe some vinegar to taste (it should taste too strong on its own). Reduce the heat to warm and carefully toss the meat in the glaze then cover and let the meat re-heat. 
                  \item While the sauce is reducing, peel, chunk up and boil your potatoes until easily pierced with a fork. Drain them off, put in a glug of milk, a big pinch of salt, and the reserved fat from the beef. Mash, stir and taste. Add more milk and salt until you like it. Chop up the fresh rosemary and stir it into the mash right before you eat.
                  \item Thaw some frozen peas and add butter and salt to taste. When you get everything on the plate, be sure to spoon some extra glaze over the beef.
            \end{enumerate}}


      \meal{Brick Chicken}
            {From Adam Ragusea }
            {2}                % Serving Size
            {\begin{itemize}      % Ingredients
                  \item 4 large boneless Chicken Thighs with skin 
                  \item 3 large Carrots 
                  \item 1 bunch Green Onions 
                  \item 2--3 cloves Garlic 
                  \item 1 red Chili
                  \item Lemon juice
                  \item Salt/Pepper 
                  \item 2 Bricks 
            \end{itemize}}
            {\begin{enumerate}    % Cooking Instructions
                  \item  Peel the carrots and discard the outer peels. Then use your peeler to cut as many thin ribbons as possible out of the carrots, discarding the core. Cut the root ends off the green onions, as well as the dry green tops, and peel off any undesirable outer layers. Bisect each onion and pull the layers apart from each other. Peel and slice/chop the garlic, and thinly slice the chili.
                  \item Season the chicken with salt and pepper. Heat a cast iron or nonstick pan on medium, and coat the bottom with olive oil. When it's shimmering, lay in the thighs, skin-side down. Place the brick (covered in foil) on top of chicken 
                  \item Reduce the heat, possibly to low. Use your nose to perceive whether the skin is burning. After about 10 minutes, remove the brick, and flip the chicken pieces to cook the raw side, about 5 additional minutes to make sure the interior is thoroughly cooked. (If the skins aren't brown enough to your liking, you could always turn up the heat right at the end and brown them some more.) Remove to a plate to rest.
                  \item Dump your vegetables into the pan, turn the heat up to medium-high, and stir constantly to get everything coated in chicken fat and cooked evenly. After about 5 minutes, the vegetables should be almost tender. Taste for seasoning, and stir in any needed salt, plus the juice of half a lemon
            \end{enumerate}}



      \meal{Chicken Karahi}
            {Pakistani dish from Ethan Chlebowski, I serve with Udon noodles}
            {5--7}                % Serving Size
            {\begin{itemize}      % Ingredients
                  \item 4--5 lbs boneless Chicken Thighs
                  \item 3--5 cloves Garlic 
                  \item 4" knob Ginger
                  \item 28oz can whole/diced Tomatoes 
                  \item 2 Tbsp Cumin 
                  \item 1 Tbsp Coriander seeds 
                  \item 2 Tbsp Garam Masala 
                  \item 1 Tbsp Kashmiri chili powder (or cayenne)
                  \item 1 Tbsp Turmeric 
                  \item MSG
                  \item Salt/Pepper 
                  \item 10 green Chiles 
                  \item 1 Tbsp Butter 
                  \item Cilantro 
                  \item Noodles, rice or roti 
            \end{itemize}}
            {\begin{enumerate}    % Cooking Instructions
                  \item  Chop the chicken, crush the garlic and half the ginger into a paste, separate tomatoes, gather all spices, chop chiles, julienne other half of the ginger, and chop cilantro leaves.
                  \item Set a wok over high heat and add the oil. Once hot, add the chicken and a generous sprinkle of salt. Cook while stirring until the chicken is cooked on the exterior.
                  \item Add the garlic and ginger paste and saute for 30 seconds. Add the whole peeled tomatoes and cook while stirring for 5--10 minutes until the tomatoes start to break down. Crush the tomatoes with a spatula to develop the sauce.
                  \item Add the spices and stir into the sauce. Let the sauce reduce for another 10--20 mins until it is fairly thick.
                  \item Toss in the chiles, and butter if using. Let the sauce reduce further until your desired consistency. It should be fairly thick and glossy. Turn off the heat, season to taste 
                  \item With the heat off, add the julienned ginger and cilantro leaves. Toss once and serve with noodles, rice or roti.
            \end{enumerate}}


      \meal{Tortellini, spinach and chicken soup}
            {Simple slowcooker meal}
            {5--8}                % Serving Size
            {\begin{itemize}      % Ingredients
                  \item 1 Onion
                  \item 3 cloves Garlic
                  \item 2 Tbsp Tomato Paste 
                  \item 1 Tbsp dried Basil
                  \item 1/3 cup Flour 
                  \item 3 Tbsp Olive Oil 
                  \item 4 cups Stock 
                  \item 28oz canned diced Tomatoes 
                  \item 2 lbs boneless Chicken Thighs 
                  \item Salt/Pepper 
                  \item 4 cups forzen cheese Tortellini 
                  \item 3 cups packed Spinach 
                  \item 1/2 cup Parmesan 
                  \item 4 cups Cream 
            \end{itemize}}
            {\begin{enumerate}    % Cooking Instructions
                  \item Sear chicken quickly in pan, remove and add to slowcooker 
                  \item  In same pan, cook onion, flour, basil, garlic, tomato paste and oil for a few minutes. Add to slowcooker 
                  \item Add broth, canned tomatoes, salt/pepper. Cook for 4--6 hours on low or 3--4 hours on high 
                  \item Remove chicken, dice on seperate cutting board. Add tortellini, spinach, parmesan and warmed cream. Return sliced chicken to pot and cook for about 10 mins on high to cook tortellini  
            \end{enumerate}}


      \meal{Pork Roulade}
            {From Adam Ragusea. Use pork loin, not tenderloin}
            {5--6}                % Serving Size
            {\begin{itemize}      % Ingredients
                  \item 3--4 lbs Pork Loin 
                  \item 1 lbs ground Sausage 
                  \item 1 Fennel 
                  \item 1 red Onion 
                  \item 1/2 cup Breadcrumbs 
                  \item Lemon (maybe two) 
                  \item 2--3 cups White Wine 
                  \item Butter 
                  \item Salt/Pepper 
                  \item Olive Oil 
            \end{itemize}}
            {\begin{enumerate}    % Cooking Instructions
                  \item  Cut the stalks off the fennel and reserve. Finely chop the bulb and get it cooking in a little olive oil in a wide pan on medium-high heat. Add diced onion. After aboutt 5 min, put in the sausage, and stir aggressively with a wooden spoon to break everything up. 
                  \item Keep cooking, stirring, and scraping the pan for about 20 minutes, until everything is very brown and the fond on the bottom of the pan is about to burn. Deglaze with just enough white wine to clean the pan. Turn off the heat, and mix in just enough breadcrumbs to soak up any loose liquid and get you a dry, crumbly texture. Mix in the zest of one lemon, and leave the stuffing to cool.
                  \item Butterfly the pork loin (I don't know how to describe this, just watch the video) then pound it out as flat as possible with a smooth meat mallet. Lay the stuffing onto the cut-side of the pork, as smooth, thin and flat as possible, leaving a small bare strip on the end of the pork that has the fat cap on the opposite side. Roll up the pork so that the fat cap is on the outside, on top. Tie with twine, place into same pan and into oven at 400F. Cover with oil, salt, pepper, add halfed lemons. Roast until internally reads 140F
                  \item Remove the roast and lemons to a cutting board and let it rest. Bring the roasting pan to a boil on high heat and deglaze with about half a bottle of white wine. Boil until it just starts to look syrupy, then turn off the heat. Mix in a knob of cold butter.
                  \item Carve the roast. Drizzle the wine jus over top, garnish with the reserved fennel fronds, and serve with the roasted lemons.
            \end{enumerate}}




      \meal{Tinga Poblana}
            {From Ethan Chlebowski, serve with tacos or rice}
            {5--7}                % Serving Size
            {\begin{itemize}      % Ingredients
                  \item 4 lbs Pork Shoulder 
                  \item 3 Bay leaves 
                  \item Salt/Pepper 
                  \item 280z--42oz cannned Tomatoes
                  \item 3 cloves Garlic 
                  \item 3--6 Chipotles in adobo (plus can liquid)
                  \item dried Oregano
                  \item pinch Sugar 
                  \item 1 Onion 
                  \item Oil 
            \end{itemize}}
            {\begin{enumerate}    % Cooking Instructions
                  \item  Cut pork into large chunks, salt and add to large pot. Cover with water, add bay leaf and bring to boil. Reduce heat to low and simmer for 45--60 mins until shreddable. Reserve 2 cups of the cooking liquid
                  \item Add tomatoes, garlic, chipotles, oregano, salt, pepper and reserved cooking liquid to blender. Blend until smooth, adjust taste with salt and sugar 
                  \item In a pan with oil, cook onions and bay leaves. Add tinga sauce and simmer 5 mins. Add shredded meat and cook 5--10 mins.
                  \item Serve over tacos or rice with veggies 
            \end{enumerate}}


      \meal{Shepard's Pie} 
            {From Gordon Ramsay}
            {5--8}                % Serving Size
            {\begin{itemize}      % Ingredients
                  \item 4 lbs ground Lamb \& Beef
                  \item 2 Carrots
                  \item 2 Onions 
                  \item frozen Peas 
                  \item Rosemary 
                  \item Thyme 
                  \item 4 cloves Garlic 
                  \item Salt/Pepper 
                  \item MSG 
                  \item Worcestershire 
                  \item 4 Tbsp Tomato Paste 
                  \item 2 cups red Wine 
                  \item 1/2 cup Stock 
                  \item 3--4 lbs Potatoes 
                  \item Milk/Cream 
                  \item Butter 
                  \item 4 Egg Yolks 
                  \item Parmesan 
            \end{itemize}}
            {\begin{enumerate}    % Cooking Instructions
                  \item  Add meat to large hot pan with oil. Brown heavily, season and add rosemary, thyme, garlic. Add grated carrots and onion. 
                  \item Add tomato paste and worcestershire (and MSG). Cook out the paste and then add wine. Cook for an additional few minutes and then add stock. Add 2--4 cups of frozen peas
                  \item Meanwhile, boil diced potatoes until tender (20 mins). Strain, mash with milk, butter, salt, pepper, garlic powder, thyme, egg yolks and parmesan 
                  \item Add mince to baking dish. Spread mash on top and level with a fork, adding lines for extra browning. Bake at 400F for 15--20 mins. Broil for 1--2 mins at the end.
                  \item Let cool for 10 mins, serve 
            \end{enumerate}}


      \meal{Meatloaf}
            {From Babish, serve with potatoes and veggies}
            {6--8}                % Serving Size
            {\begin{itemize}      % Ingredients
                  \item 1.5 cup Buttermilk 
                  \item 15oz Breadcrumbs
                  \item 4 lbs ground Beef/Pork/Veal 
                  \item 2 Tbsp Garlic Powder 
                  \item 1 large Onion, grated 
                  \item 3 stalks Celery, grated 
                  \item 3 Carrots, grated 
                  \item 1 cup Tomato Paste 
                  \item 1/2 cup fresh Parsley, chopped 
                  \item 1/4 cup fresh Basil, chopped 
                  \item 6 large Eggs 
                  \item 4 Tbsp Worcestershire
                  \item Salt/Pepper/MSG 
                  \item 1.5 cup Ketchup
                  \item 1/3 cup Brown Sugar 
                  \item 3 Tbsp cider Vinegar
            \end{itemize}}
            {\begin{enumerate}    % Cooking Instructions
                  \item  Prepare a small rimmed baking sheet with tin foil and an even layer of nonstick spray and prepare a 9-inch loaf pan with a light coating of nonstick spray.
                  \item Combine the buttermilk, bread crumbs, and garlic powder in large bowl. Add the vegetables with the meatloaf mix, 3/4 cup tomato paste, parsley, basil, eggs, 1 Tbps Worcestershire sauce, salt, and 1 teaspoon of pepper. Fold mixture by hand 
                  \item Press the meatloaf mixture into the prepared loaf pan. Flip the meatloaf out onto the foil-lined baking sheet
                  \item Meanwhile, prepare the glaze by combining the ketchup, tomato paste, brown sugar, honey, vinegar, Worcestershire sauce, and remaining pepper in a small bowl. Whisk to combine. 
                  \item Bake the meatloaf for 15 mins at 325F, then brush it with about half of the glaze. Brush every 15 mins , until an internal temperature of 155F is reached (30--45 mins)
                  \item Remove from oven and cool for 10 mins before slicing 
            \end{enumerate}}


      \meal{Tuscan Chicken}
            {Lazy creamy slowcooker meal}
            {5--7}                % Serving Size
            {\begin{itemize}      % Ingredients
                  \item 4--6 Chicken Breast 
                  \item 1--2 large Onions 
                  \item 4 cloves Garlic 
                  \item 2 Tbsp Italian Seasoning 
                  \item 1 tps Red Pepper Flakes 
                  \item 15--24oz Alfredo Sauce 
                  \item 1 cup Parmesan 
                  \item 4 cups fresh Spinach
                  \item 8oz sun dried Tomatoes
                  \item Butter 
                  \item Salt/Pepper
                  \item Pasta
            \end{itemize}}
            {\begin{enumerate}    % Cooking Instructions
                  \item  Heat a skillet on med-high heat and add the butter. Add the chicken breasts and brown both sides. Remove chicken and place in the slow cooker.
                  \item Add the onions to the skillet and cook until just turning translucent. Add the garlic and sun dried tomatoes to the onions and cook for 2 minutes, stirring occasionally. Add mixture to slowcooker 
                  \item Sprinkle in the Italian seasoning and red pepper flakes. Pour the Alfredo sauce over the top and close the lid. Cook on Low for 3--4 hours or High for 2--3 hours.
                  \item  When cook time is finished, open lid and stir in parmesan, then stir in the spinach and close the lid. Let cook for a few more minutes, until spinach is wilted/softened. Add salt/pepper to taste 
                  \item Serve over pasta with side veggies 
            \end{enumerate}}




      \meal{Gumbo}
            {Chicken and sausage Gumbo. Cook the roux as dark as possible, do not walk away and burn it. From Isaac Toups}
            {6--8}                % Serving Size
            {\begin{itemize}      % Ingredients
                  \item 1 lbs Andouille sausage, sliced
                  \item 4 lbs boneless Chicken thighs
                  \item Scallions
                  \item Cumin
                  \item MSG
                  \item Salt/Pepper
                  \item Cayenne
                  \item smoked Paprika
                  \item 1 cup Vegetable oil
                  \item 2--3 Okra 
                  \item 1 cup Flour
                  \item 1 bag Celery
                  \item 3--4 Bell Peppers
                  \item 2--3 large Onions
                  \item Bay leaves
                  \item 5--8 cloves Garlic, crushed
                  \item 1 dark Beer 
                  \item 4 cups Stock 
                  \item cooked White Rice (3 cups dry)
            \end{itemize}}
            {\begin{enumerate}    % Cooking Instructions
                  \item Dice holy trinity (onion, peppers, celery), heat 1 cup oil on medium high. Add flour, mix well. Whisk continuously until the roux is the color of a chocolate bar (10--20 mins). 
                  \item Add holy trinity and bay leaves. Cook for a few mins, then add garlic. Then slowly whisk in beer and 4 cups stock. Bring to slow boil and reduce to simmer. Add cayenne, pepper, paprika, cumin, msg and sliced andouille.
                  \item  In seperate large pan over high heat, brown chicken hard on both sides. Add to pot, deglaze pan as needed with stock. Add sliced Okra.
                  \item Simmer on low for a minimum of 45 mins, more if possible. Season to taste 
                  \item Serve over rice, garnish with green onion.
            \end{enumerate}}


      \meal{Potato Hash}
            {From J Kenji Lopez-Alt. Bonus points for corned beef}
            {5}                % Serving Size
            {\begin{itemize}      % Ingredients
                  \item 4 lbs Potatoes
                  \item 1 lbs Adouille sausage
                  \item 1--2 Onions 
                  \item 2 Bell Peppers 
                  \item Chives 
                  \item 2 Avacados 
                  \item 6--8 Eggs 
                  \item Salt/Pepper
                  \item Hot Sauce 
                  \item Oil
            \end{itemize}}
            {\begin{enumerate}    % Cooking Instructions
                  \item Dice potatoes into slightly less than 1" cubes, add to salted boiling water. Parcook for 5 mins, drain and set aside 
                  \item Meanwhile in large pan, heat oil over medium high. Fry sliced sausage until browned, set aside in bowl. 
                  \item Dice onions and peppers, cook in pan until soft, add to bowl with sausage.
                  \item Return the large pan to heat with oil, add potatoes and brown on all sides without breaking them up (maybe 20 mins).
                  \item Return vegetables and sausage to pan, add salt/pepper/hotsauce. Mix. 
                  \item Crack eggs into wells within the hash. Transfer pan to 400F oven, cook until whites set (3--5 mins)
                  \item Dice avacado and chives, garnish and serve 
            \end{enumerate}}
            

      \meal{Mississippi Pot Roast}
            {Dead simple, serve over rice with veggies or in a roll}
            {5--8}                % Serving Size
            {\begin{itemize}      % Ingredients
                  \item 4--6 lbs Pork butt/shoulder
                  \item 1 package Ranch powder 
                  \item 1 package Au Jus powder 
                  \item 1 stick salted Butter 
                  \item 1/2 jar (16 oz) sliced Pepperoncini plus juice 
                  \item Salt/Pepper 
            \end{itemize}}
            {\begin{enumerate}    % Cooking Instructions
                  \item  Add meat to slowcooker, slice peppers as needed 
                  \item Add powders, sliced butter, peppers and juice on top of meat
                  \item Cook 8--10 hours on low, shred. Optionally spread on baking sheet and broil, then return to juices
                  \item Serve over rice with veggies or as a sandwhich
            \end{enumerate}}



      \meal{Chicken Tikka Masala}
            {From J Kenji Lopez Alt, can serve with peas. A more authentic and involved recipe found \href{https://www.vice.com/en/article/v7g899/chicken-tikka-masala-recipe}{here} }
            {4--6}                % Serving Size
            {\begin{itemize}      % Ingredients
                  \item 5 lbs skinless Chicken Thighs
                  \item 3 Tbsp Cumin
                  \item 3 Tbsp Paprika
                  \item 2 Tbsp ground Coriander
                  \item 2 tsp Tumeric
                  \item 1 Tbsp Cayenne
                  \item 12 cloves Garlic
                  \item 3" fresh Ginger 
                  \item 2 cups Yogurt 
                  \item 4--6 Lemons 
                  \item Salt/Pepper 
                  \item 4 Tbsp Ghee/Butter 
                  \item 2 Onions 
                  \item 28oz crushed Tomato 
                  \item Cilantro
                  \item 1--2 cups Heavy Cream 
            \end{itemize}}
            {\begin{enumerate}    % Cooking Instructions
                  \item  Combine cumin, paprika, coriander, turmeric, and cayenne in a small bowl and mix well. Set aside 3 tablespoons of spice mixture. Combine remaining 6 tablespoons spice mixture, 8 cloves garlic, 2 tablespoons ginger, yogurt, 1/2 cup lemon juice, and 1/4 cup salt in a large bowl and whisk to combine. Pour marinade all over chicken pieces, using hands to coat every surface. Marinade at least 4 hours (ideally overnight)
                  \item Heat butter or ghee in a large Dutch oven over medium-high heat until melted and foaming subsides. Add onions, remaining 4 tablespoons grated garlic, and remaining 2 tablespoons ginger. Cook, stirring frequently, until dark and beginning to char in spots, about 10 minutes. Add reserved spice mixture and cook, stirring frequently, until fragrant, about 30 seconds. Add tomatoes and half of cilantro, scraping up any browned bits from bottom of pan with a spoon. Simmer for 15 minutes, then puree using a hand blender or by transferring to a tabletop blender in batches.
                  \item Stir in cream and remaining quarter cup lemon juice. Season to taste with salt, then set aside until chicken is cooked.
                  \item Grill chicken on high heat for 10--12 mins total, to get good color/charr but not completely cook. Chop as desired
                  \item Transfer chicken chunks to pot of sauce. Bring to a simmer over medium heat and cook, stirring frequently, until chicken is just cooked through, about 10 minutes. Sprinkle with remaining cilantro, then serve immediately with rice or grilled naan 
            \end{enumerate}}

      \meal{Chipotle Bowl}
            {Grilled chicken with cilantro-lime rice and fiesta veggies}
            {6--7}                % Serving Size
            {\begin{itemize}      % Ingredients
                  \item 5 lbs chicken Thighs
                  \item 3--4 cups white rice (dry)
                  \item 3--4 Limes
                  \item 14--28 oz black Beans
                  \item 3--4 cups frozen Corn 
                  \item 2--3 Onions
                  \item 3--4 Poblano Peppers 
                  \item 1--2 cups heavy Cream 
                  \item 2 Tbsp Chili Powder 
                  \item 2 Tbsp Cumin 
                  \item 2 Tbsp smoked Paprika 
                  \item 1 Tbsp Cayenne
                  \item Cilantro 
                  \item 3--5 cloves Garlic 
                  \item Achiote Paste or 3 Sazon seasoning packets 
                  \item Oil/Mayo 
                  \item Salt/Pepper 
            \end{itemize}}
            {\begin{enumerate}    % Cooking Instructions
                  \item  Make chicken marinade in blender by adding half an onion, 2 habaneros, 1 lime zest + juice, 2 cloves garlic, Achiote or sazon, saltand oil/Mayo. Adjust thickness with water or more fat. Marniade chicken for at least one hour.
                  \item Cook rice in rice cooker. Add salt, washed black beans, chopped cilantro and zest + juice from remaining limes. 
                  \item In large dry pan over high heat, add corn until charred. Then add onion, peppers, garlic, salt and oil. Cook until soft, then add cumin, chili powder, cayenne and paprika to taste. After a few minutes add heavy cream until rich and a thick sauce coats the veggies. 
                  \item Grill and chop chicken. Serve in bowl with equal parts rice and veggies 
            \end{enumerate}}

      \meal{Feijoada}
            {Brazillian black-bean poverty stew served over rice with orange slices. Designed for any cheap off cuts of pork (trotters, necks, ears ect)}
            {6--9}                % Serving Size
            {\begin{itemize}      % Ingredients
                  \item 1 lbs dry Black Beans
                  \item 1 lbs Pork Shoilder
                  \item 1 lbs dried (or corned or stew) Beef
                  \item 1--2 lbs Pork Neck
                  \item 2 smoked Ham Hocks 
                  \item 1 lbs smoked Chorizo 
                  \item 1/2 lbs Bacon
                  \item 3--5 Bay leaves 
                  \item Salt/Pepper 
                  \item 3 Onions 
                  \item 3 large Tomatoes
                  \item Cilantro
                  \item 2--4 bell Peppers 
                  \item 2--3 Oranges 
                  \item Kale
                  \item Cayenne/Hot sauce 
                  \item Rice 
            \end{itemize}}
            {\begin{enumerate}    % Cooking Instructions
                  \item  Soak beans overnight, discard water. Add to largest pot and cover with 2" water. Add salt, bay leaves and chopped cilantro. Bring to a simmer
                  \item Meanwhile, slice and slowly render bacon in seperate pan. When crisp, add to pot and set fat aside for cooking veggies 
                  \item In batches, brown all meats (cut into 2" sized chunks) in pan and then transfer to pot. Add additional water as needed. 
                  \item Cook diced onions and peppers in pan until soft, scraping any fonde from the bottom of the pan. Add to pot with diced tomatoes and cayenne or hot sauce
                  \item Simmer for 5--8 hours, add water as necessary. Remove bones. Adjust seasoning with salt and acid. Then liquid should turn a deep brown and thicken slightly from the gelatin dissolved from the pork parts
                  \item In last 30 mins of cooking, add kale to pot (washed, chopped and large stems removed)
                  \item Serve over rice with orange slices. Excellent when reheated as leftovers 
            \end{enumerate}}

      \meal{Fried Rice (cast iron method)}
            {Using a wok would be better}
            {5--8}                % Serving Size
            {\begin{itemize}      % Ingredients
                  \item 5 (dry) cups of cooked white rice
                  \item 6--8 eggs 
                  \item frozen Peas 
                  \item Seasme Oil 
                  \item Vegetable Oil 
                  \item Oyster Sauce 
                  \item Soy Sauce 
                  \item Sriracha 
                  \item Green onion 
                  \item Sesame Seeds 
                  \item MSG 
                  \item 2--3 Onions 
                  \item 2--3 Bell Peppers 
                  \item 2--4 lbs Pork chops 
            \end{itemize}}
            {\begin{enumerate}    % Cooking Instructions
                  \item  Cook rice the night before, leave in fridge 
                  \item Sear the pork chops in pan until cooked and browned. Chop into small pieces
                  \item Working in batches, add 1 Tbps oil to pan, cook dice onions and peppers until soft. Then push to the side and add more oil, then add rice in even layer. Heat should be high to crisp some of the rice 
                  \item Scrape and stir rice until desired amount of fried. Add oil as necessary and keep heat high. Add scrambled eggs and cook fully before chopping and mixing with rice and veggies 
                  \item Return pork to pan, add peas and season to taste with soy sauce, oyster sauce, seasme oil, sriracha and MSG 
                  \item Garnish with seasme seeds and green onion. Serve warm 
      \end{enumerate}}     

      \meal{Corn Chowder}
            {From Neuman family, MUST be served with cornbread}
            {5--6}                % Serving Size
            {\begin{itemize}      % Ingredients
                  \item 1 lbs Bacon
                  \item 2 large Onions
                  \item 3 bell Peppers 
                  \item 2 lbs Ham (steak)
                  \item 4 cups frozen corn 
                  \item 2--4 ears corn 
                  \item 28 oz diced Tomatoes
                  \item 1 Tbsp Oregano 
                  \item 1 Tbsp Basil 
                  \item 1 Tbsp Thyme 
                  \item 1 Tbsp Red Pepper 
                  \item Black Pepper (lots) 
                  \item 4--8 cups Milk 
                  \item 1 cup Flour 
                  \item Vegetable Oil 
                  \item 2 cups heavy Cream 
            \end{itemize}}
            {\begin{enumerate}    % Cooking Instructions
                  \item  Cook bacon in large pot until crisp, remove and leave fat behind. Add diced ham-steak and sear. Remove add meat from pot
                  \item Cook diced onion and peppers until tender. Add corn (including peeled and scraped ears) to pot. Then add seasonings and tomatoes (and maybe a bit of water/stock) and simmer for 30 mins. A
                  \item Meanwhile, cook flour and oil to make a roux. Then slowly add milk to make a bechamel sauce, bring to a simmer to thicken. 
                  \item Scoop some amount of the veggie mixture into blender and blend smooth (amount depends on desired final texture). 
                  \item Add everything to large pot and simmer for at least 30 mins
                  \item Adjust seasoning to taste, should be heavy on the black pepper. Serve with cornbread and hotsauce. 
      \end{enumerate}}

      \printindex

%%%%%%%%%%%%%%%%%%%%%%%%%%%%%%%%%%%%%%%%
      % \meal{RecipeTitle}
      %       {OptionalComment}
      %       {5--8}                % Serving Size
      %       {\begin{itemize}      % Ingredients
      %             \item 
      %       \end{itemize}}
      %       {\begin{enumerate}    % Cooking Instructions
      %             \item  
      %       \end{enumerate}}
%%%%%%%%%%%%%%%%%%%%%%%%%%%%%%%%%%%%%%%%

\end{document}